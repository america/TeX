\documentclass[fleqn,leqno,autodetect-engine,dvipdfmxi-if-dvi,ja=standard]{bxjsarticle}
\begin{document}

オイラーの定理\\

nが正の整数でaをnと互いに素な正の整数としたとき、
\begin{displaymath}
a^{\varphi (n)} \equiv 1 \pmod{n}
\end{displaymath}
が成立する。ここで$\varphi (n)$はオイラーの$ \phi $関数である。\\

この定理はフェルマーの小定理の一般化であり、この定理をさらに一般化したものがカーマイケルの定理である。

\end{document}


