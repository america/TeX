\documentclass[fleqn,leqno,autodetect-engine,dvipdfmx-if-dvi,ja=standard]{bxjsarticle}
\begin{document}
\part{おいしいコンパーニュ}
ここでは、おいしいコンパーニュについて解説します.

\section{クリストンのやり方}
グルジニアヨーグルトを使い、検便虫を栽培します。鴛海先生。ああ、ギストたちよ。ギフト。キャロンッ!似てねー。ジミー寺西。ナイト・サタデー。グルジアワイン。もう何もわからない。俺は多分もうすぐ死ぬ。何か残せたらと思うんだけど、俺なんにもないんだよね。\\

何か残せたらよかったんだけど・・・

\section{エンシントン・エネルギー}
グリーフ・ハムをやりながら、インド人たちは向かう。{\Large ああ、幸せとは?}「生まれたからって幸せになれなきゃ苦しいだけじゃねえか。」とアンドロイド・コスタリカは言う。うおお、苦しい。ペッツ。フォゥー!俺の名前はツイダック。電気が体を走り抜けていく。ライスカレーくれや。Come on, baby. You gatta know why.丸いからって四角いとは限らないだろう?あー、焼き鳥食いてえ。

\section{クリピストン・ガット・フォー・ザ・モーメント}
おお、チョブリスよ。出たー、丸がし消しゴム。よーし、メルボルンに着いたぞ。うーん、書くことが思い浮かばねえな。いいもんだろうさ、金は。今夜はゆっくりできるんだろう?ゴリラみたいな顔しやがって。おい、クンドルホ、俺達の分もあるんだろうな?あーあ、だりい。ゴムボール。

\section{ラプソディ・イン・ブルー}
書くことが生きる事だと信じて書いている。{\large \textgt{ゴッド・ブレス・ユー!}}「なんて綺麗なワイシャツなんだ。ちきしょう、涙が出てきちまう。」レイチェルはワイシャツの山に顔を押し付け、声を押し殺してむせび泣いた。なんだ、この底の浅い文章は?カンヴァスに野菜を描いてみよう。けりが入る。俺の名はロボ・トム・ボーイ。東京ドームという大舞台。くせーんだよ。ゴミ野郎。口が生ごみ臭い。センブリティ。みんなエネルギーに溢れてるんだな。母さんは俺の何を知っていたの?俺が知らないこと?そういえば、俺は将来自分が何かできるって思ってたな。そう思わなくなったのはいつからだろう?

\section{順列}

n個から順序を意識してk個取り出す場合の数(順列の定義)
\[ {}_n P _k  =  \frac{n!}{(n-k)!} \]

\section{組み合わせ}

n個から順序を意識せずにk個取り出す場合の数(組み合わせの定義)
\[ {}_n C _k = \left(
               \begin{array}{c}
                 n \\
                 k 
	       \end{array}
               \right) = \frac{n!}{k!(n-k)!}  
\]
\end{document}